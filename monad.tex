\begin{frame}[fragile]
\frametitle{Monad}
\begin{block}{A monad F is given by the function:}
\begin{lstlisting}
    (a -> F b) -> (F a -> F b)
\end{lstlisting}
\end{block}
\emph{binds a function through an environment F}
\end{frame}

\begin{frame}[fragile]
\frametitle{Functor vs Monad}
\begin{block}{Functor vs Monad}
\begin{lstlisting}
    (a ->   b) -> (F a -> F b)
    (a -> F b) -> (F a -> F b)
\end{lstlisting}
\end{block}
\end{frame}

\begin{frame}[fragile]
\frametitle{Monad Haskell}
\begin{block}{Monad using Haskell syntax}
\begin{lstlisting}[language=haskell]
    class Monad f where
      bind :: (a -> f b) -> (f a -> f b)
\end{lstlisting}
\end{block}
\end{frame}

\begin{frame}[fragile]
\frametitle{Monad Examples}
\begin{block}{Examples of Monads}
\begin{itemize}
  \item The list monad takes the cartesian product:
    \begin{lstlisting}
    (a -> [] b) -> ([] a -> [] b)
    (a -> [b]) -> ([a] -> [b])
    \end{lstlisting}
  \item The maybe monad threads the possible Nothing:
    \begin{lstlisting}
    (a -> Maybe b) -> (Maybe a -> Maybe b)
    \end{lstlisting}    
\end{itemize}
\end{block}
\end{frame}

\begin{frame}[fragile]
\frametitle{Monad Examples in Haskell}
\begin{block}{Examples of Monads using Haskell syntax}
\begin{lstlisting}[language=haskell]
    instance Monad [] where
      bind f =
        foldr ((++) . f) []

    instance Monad Maybe where
      bind f =
        maybe Nothing f

    instance Monad ((->) t) where
      bind f g =
        \x -> f (g x) x
\end{lstlisting}
\end{block}
\end{frame}
