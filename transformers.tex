\begin{frame}[fragile]
\frametitle{Maybe Monad Transformer}
And so the Maybe monad transformer comes to be.

\begin{lstlisting}[language=haskell]
  data MaybeT f a = MaybeT { 
    maybeT :: f (Maybe a)
  }

  instance Monad f => Monad (MaybeT f) where
    bind f (MaybeT x) =
      MaybeT 
        (bind 
          (maybe (unit Nothing) 
          (maybeT . f)) x
        )
\end{lstlisting}
\end{frame}

\begin{frame}[fragile]
\frametitle{Maybe Monad Transformer}
\begin{block}{The Maybe monad transformer}
provides the construction of the monad for (f of Maybe) for an arbitrary monad f. Its behaviour combines the individual monads of Maybe then f, in that order.
\end{block}
\begin{block}{This transformer exists}
precisely because \emph{Monads do not compose in general}.
\end{block}
\end{frame}

\begin{frame}[fragile]
\frametitle{Maybe Monad Transformer}
\begin{block}{Example [] on Maybe}
\begin{lstlisting}[language=haskell]
m1 :: MaybeT [] Integer
m1 = MaybeT [Just 1, Just 2, Just 30]

f1 :: Integer -> MaybeT [] Integer
f1 n =
  MaybeT
    [
      Just n
    , if n < 10 then Just (n * 50) else Nothing
    ]
\end{lstlisting}
\end{block}
\begin{lstlisting}
> maybeT (bind f1 m1)
[Just 1,Just 50,Just 2,Just 100,Just 30,Nothing]
\end{lstlisting}
\end{frame}

\begin{frame}[fragile]
\frametitle{Maybe Monad Transformer}
\begin{block}{Example ((->) t) on Maybe}
\begin{lstlisting}[language=haskell]
m2 :: MaybeT ((->) Integer) String
m2 = MaybeT (\x ->
       if even x
         then Just (show (x * 10))
         else Nothing)

f2 :: String -> MaybeT ((->) Integer) String
f2 s = MaybeT (\n -> 
         if n < 100
           then Just (show n ++ s)
           else Nothing)
\end{lstlisting}
\end{block}
\begin{lstlisting}
> map (maybeT (bind f2 m2)) [3, 4, 700]
[Nothing,Just "440",Nothing]
\end{lstlisting}
\end{frame}

\begin{frame}[fragile]
\frametitle{More Monad Transformers}
\begin{itemize}
  \item \lstinline $MaybeT f a = f (Maybe a)$
  \item \lstinline $EitherT f a b = f (Either a b)$
  \item \lstinline $ReaderT f a b = a -> f b$
  \item \lstinline $StateT f s a = a -> f (a, s)$
\end{itemize}
\begin{block}{Each Monad Transformer exists}
because \emph{monads do not compose in general}
\end{block}
\end{frame}

\begin{frame}[fragile]
\frametitle{Transformers}
\begin{block}{Functor Transformers do not exist}
because \emph{functors compose} so what's the point?
\end{block}
\end{frame}

\begin{frame}[fragile]
\frametitle{Transformers}
\begin{block}{Actually, you'll find all these things compose:}
\begin{itemize}
\item Functor
\item Apply
\item Applicative
\item Alt
\item Alternative
\item Foldable
\item Foldable1
\item Traversable
\item Traversable1
\end{itemize}
\end{block}
\end{frame}

\begin{frame}[fragile]
\frametitle{Transformers}
\begin{block}{These do not compose:}
\begin{itemize}
\item Monad
\item Bind
\item Comonad
\item Cobind
\end{itemize}
\emph{it's a useful exercise to try it anyway!}
\end{block}
\end{frame}

\begin{frame}[fragile]
\frametitle{Questions}
?
\end{frame}
